\documentclass[12pt]{article}

\usepackage[utf8]{inputenc}
\usepackage[T2A]{fontenc}
\usepackage[english, russian]{babel}

\usepackage{amsmath, amsfonts, amsthm, amssymb, amsopn, amscd}
\usepackage{enumerate, bm}
\usepackage[mathscr]{eucal}

\usepackage{hyperref}
\hypersetup{unicode=true,final=true,colorlinks=true}

\theoremstyle{remark}
%\newtheorem{exercise}{Упражнение}
\newtheorem{exercise}{}[subsection]
\renewcommand{\theexercise}{\textbf{\#\arabic{exercise}}}

%
%   Вероятностные определения
%
\DeclareMathOperator{\cov}{cov}
\DeclareMathOperator{\corr}{corr}
\DeclareMathOperator*{\plim}{plim}
\DeclareMathOperator{\Var}{Var}
\DeclareMathOperator{\VVar}{V}
\DeclareMathOperator{\sVar}{sVar}

%
%  Суммы квадратов
%
\DeclareMathOperator{\TSS}{TSS}
\DeclareMathOperator{\RSS}{RSS}
\DeclareMathOperator{\ESS}{ESS}

%
%  Эконометрические
%
\newcommand{\resid}{{\hat{u}}}
\DeclareMathOperator{\const}{const}
\DeclareMathOperator{\error}{error}
\DeclareMathOperator{\StdError}{s.e.}
\DeclareMathOperator{\HCStdError}{HC.s.e.}
\DeclareMathOperator{\HACStdError}{HAC.s.e.}
\DeclareMathOperator{\SER}{SER}
\DeclareMathOperator{\DW}{DW}
\DeclareMathOperator{\probit}{probit}
\DeclareMathOperator{\logit}{logit}
\DeclareMathOperator{\gompit}{gompit}
\DeclareMathOperator{\loglog}{loglog}
\DeclareMathOperator{\WhiteNoise}{{WN}}
\DeclareMathOperator{\DurbinWatson}{DW}
\DeclareMathOperator{\VAR}{VAR}
\DeclareMathOperator{\ARMA}{ARMA}
\DeclareMathOperator{\ARIMA}{ARIMA}

%
%  Линейная алгебра
%
\DeclareMathOperator{\rank}{rank}
\DeclareMathOperator{\dimension}{dim}
\newcommand{\LinearSpace}{{\mathfrak L}}


%
%   Числовые
%
\newcommand{\Complex}{{\mathbb C}}
\newcommand{\N}{\mathbb N}
\newcommand{\Z}{{\mathbb Z}}
\newcommand{\Q}{{\mathbb Q}}
\newcommand{\R}{{\mathbb R}}
\newcommand{\semiaxes}{{\mathbb R_+}}

%
%  Вероятностные
%
\newcommand{\iid}{{i.i.d.}}
\newcommand{\Exp}{{\mathsf E}}
\newcommand{\Gauss}{{\mathscr N}}
\newcommand{\Likelihood}{{\mathcal L}}
\newcommand{\StError}{{s.e.}}
\newcommand{\ConfInterval}{{\mathcal I}}

%
%   Вектора
%
\newcommand{\vconst}{{\mathbf const}}
\newcommand{\vectx}{{\bm x}}
\newcommand{\vecty}{{\bm y}}
\newcommand{\vectz}{{\bm z}}
\newcommand{\vecte}{{\bm e}}
\newcommand{\vectw}{{\bm w}}
\newcommand{\vecth}{{\bm h}}
\newcommand{\vectr}{{\bm r}}
\newcommand{\vectq}{{\bm q}}
\newcommand{\vectf}{{\bm f}}%{\boldsymbol{f}}
\newcommand{\vectu}{{\bm u}}
\newcommand{\vectv}{{\bm v}}
\newcommand{\vectalpha}{{\bm{\alpha}}}
\newcommand{\vectbeta}{{\bm{\beta}}}
\newcommand{\vectgamma}{{\bm{\gamma}}}
\newcommand{\vectdelta}{{\bm{\delta}}}
\newcommand{\vecteta}{{\bm{\eta}}}
\newcommand{\vectpi}{{\bm{\pi}}}
\newcommand{\vectmu}{{\bm{\mu}}}

%
%  Матрицы
%
\newcommand{\Id}{I}
\newcommand{\matrixX}{{\bm X}}
\newcommand{\matrixY}{{\bm Y}}
\newcommand{\matrixU}{{\bm U}}
\newcommand{\matrixV}{{\bm V}}
\newcommand{\matrixR}{{\bm R}}
\newcommand{\matrixZ}{{\bm Z}}
\newcommand{\matrixA}{{\bm A}}
\newcommand{\matrixB}{{\bm B}}
\newcommand{\matrixQ}{{\bm Q}}
\newcommand{\matrixH}{{\bm H}}
\newcommand{\matrixGamma}{{\bm{\Gamma}}}
\newcommand{\matrixPi}{{\bm{\Pi}}}

%
% Теоремы, Примеры etc
%

\theoremstyle{plain}
\newtheorem*{teorema}{Теорема}
\newtheorem*{importante}{Важно!}
\newtheorem*{ejemplo}{Пример}
\newtheorem*{definicion}{Определение}

\theoremstyle{remark}
\newtheorem*{remark}{Замечание}

%
%  Язык R
%
\newcommand{\Rlang}{\texttt{R}}


\title{Задачи по Эконометрике временных рядов}

\author{Н.В. Артамонов}

\begin{document}

\maketitle

\tableofcontents

\section{Работа с рядами в Python. Визуализация}

\begin{exercise}
Из БД FRED загрузите \textbf{квартальные} данные по ВВП США с 1990 Q1 по н.в. (ряд \(gdp\))
\begin{enumerate}
	\item Задайте квартальный временной индекс
	\item Визуализируйте ряд \(gdp\)
	\item Визуализируйте ряд \(\log(gdp)\)
	\item Визуализируйте ряд \(\Delta\log(gdp)\)
	\item Визуализируйте ряд \(\Delta^2\log(gdp)\)
	\item Постройте гистограммы для \(\log(gdp),\Delta\log(gdp),\Delta^2\log(gdp)\)
	\item Постройте диаграмму рассеяние \(\log(gdp_t)\) vs \(\log(gdp_{t-1})\)
	\item Постройте диаграмму рассеяние \(\Delta\log(gdp_t)\) vs \(\Delta\log(gdp_{t-1})\)
	\item вычислите \(\corr(\log(gdp_t), \log(gdp_{t-1}))\) и 
	тестируйте его значимость (формально!)
	\item вычислите \(\corr(\Delta\log(gdp_t), \Delta\log(gdp_{t-1}))\) и 
	тестируйте его значимость (формально!) 
\end{enumerate}
\end{exercise}

\begin{exercise}
Из БД FRED загрузите \textbf{месячные} данные по M2 США с 1990-01-01 по н.в. (ряд \(m2\))
\begin{enumerate}
	\item Задайте месячный временной индекс
	\item Визуализируйте ряд \(m2\)
	\item Визуализируйте ряд \(\log(m2)\)
	\item Визуализируйте ряд \(\Delta\log(m2)\)
	\item Визуализируйте ряд \(\Delta^2\log(m2)\)
	\item Постройте гистограммы для \(\log(m2),\Delta\log(m2),\Delta^2\log(m2)\)
	\item Постройте диаграмму рассеяние \(\log(m2_t)\) vs \(\log(m2_{t-1})\)
	\item Постройте диаграмму рассеяние \(\Delta\log(m2_t)\) vs \(\Delta\log(m2_{t-1})\)
	\item вычислите \(\corr(\log(m2_t), \log(m2_{t-1}))\) и 
	тестируйте его значимость (формально!)
	\item вычислите \(\corr(\Delta\log(m2_t), \Delta\log(m2_{t-1}))\) и 
	тестируйте его значимость (формально!)
\end{enumerate}
\end{exercise}

\begin{exercise}
Из БД FRED загрузите \textbf{недельные} данные по M2 США с 1990-01-01 по н.в.
\begin{enumerate}
	\item агрегируйте их в квартальные наблюдения (через усреднение)
	\item задайте квартальный временной индекс
	\item визуализируйте полученные наблюдения 
\end{enumerate}
\end{exercise}

\begin{exercise}
Из БД FRED загрузите месячные данные краткосрочной (3-х мес, \(rate1\)) и долгосрочной (10-ти лет., \(rate2\)))
ставкам для США с 1990-01-01 по н.в. как многомерный временной ряд \(rates\).
\begin{enumerate}
	\item Задайте месячный временной индекс
	\item Визуализируйте ряд \(rates\) двумя способами
	\begin{itemize}
		\item раздельные графики
		\item общий график (два ряда на одном графике)
	\end{itemize}
	\item Визуализируйте ряд \(\Delta\log(rates)\) двумя способами
	\item Визуализируйте ряд \(\Delta^2\log(rates)\) двумя способами
	\item Постройте гистограммы для \(rates,\Delta rates,\Delta^2 rates\) двумя способами
	\item Постройте диаграмму рассеяние \(rate1\) vs \(rate2\)
	\item Постройте диаграмму рассеяние \(\Delta rate1 \) vs \(\Delta rate2\)
	\item вычислите \(\corr(rate1, rate2)\) и 
	проверьте его значимость (формально!)
	\item вычислите \(\corr(\Delta rate1, \Delta rate2)\) и 
	проверьте его значимость (формально!)
\end{enumerate}
\end{exercise}

\begin{exercise}
Из БД FRED загрузите месячные данные по США
\begin{itemize}
	\item краткосрочная (3-х мес) ставка
	\item долгосрочная (10-ти лет) ставка
	\item логарифм денежной массы M2
\end{itemize}
с 2000-01-01 по н.в. как многомерный временной ряд
\begin{enumerate}
	\item задайте месячный временной индекс
	\item Визуализируйте многомерный ряд
	\item Визуализируйте первую и вторую разность
	\item Вычислите корреляционную матрицу для исходного ряда и визуализируйте её
	\item Вычислите корреляционную матрицу для дифференцированного ряда и визуализируйте её
\end{enumerate}
\end{exercise}

\begin{exercise}
Из finance.yahoo.com загрузите данные c 2005-01-01 по 2023-12-31 по S\&P500
\begin{enumerate}
	\item Сформируйте месячный временной ряд из цены закрытия на последний день каждого месяца
	\item Задайте для него месячный временной индекс
	\item Визуализируйте ряд
	\item Визуализируйте первую и вторую логарифмические разности
\end{enumerate}
\end{exercise}

\begin{exercise}
Из finance.yahoo.com загрузите данные c 2005-01-01 по 2023-12-31 по
ценам закрытия S\&P500, Apple, Google
\begin{enumerate}
	\item Сформируйте многомерный ряд из цен закрытия на последний день каждого месяца
	\item Визуализируйте многомерный ряд
	\item Визуализируйте первую и вторую логарифмические разности
	\item Вычислите корреляционную матрицу для исходного ряда и визуализируйте её
	\item Вычислите корреляционную матрицу для лог-разности ряда и визуализируйте её
\end{enumerate}
\end{exercise}

\section{ACF \& PACF}

Во всех задачах по умолчанию уровень значимости 5\%.

\begin{exercise}
Рассмотрим квартальные ряды
\begin{itemize}
	\item \(x\): первая разность логарифма ВВП США с 1990 Q1 по н.в.
	\item \(y\): вторая разность логарифма ВВП США с 1990 Q1 по н.в.
\end{itemize}
Для них
\begin{enumerate}
	\item Постройте график ряда, ACF и PACF для каждого ряда
	\item Значимы ли \(r(3),r_{part}(3)\)?
	% \item Тестируйте совместную значимость первых 5 коэффициентов ACF.
\end{enumerate}
\end{exercise}

\begin{exercise}
Рассмотрим месячные ряды
\begin{itemize}
	\item \(x\): первая разность 3-х месячной ставки США с 2000-01 по
	н.в.
	\item \(y\): вторая разность 3-х месячной ставки США с 2000-01 по
	н.в.
\end{itemize}
Для них
\begin{enumerate}
	\item Постройте график ряда, ACF и PACF для каждого ряда
	\item Значимы ли \(r(4),r_{part}(4)\)?
	% \item Тестируйте совместную значимость первых 6 коэффициентов ACF.
\end{enumerate}
\end{exercise}

\begin{exercise}
Рассмотрим месячные ряды
\begin{itemize}
	\item \(x\): первая разность логарифма S\&P500 с 2000-01 по
	н.в.
	\item \(y\): вторая разность логарифма S\&P500 с 2000-01 по
	н.в.
\end{itemize}
Для них
\begin{enumerate}
	\item Постройте график ряда, ACF и PACF для каждого ряда
	\item Значимы ли \(r(5),r_{part}(5)\)?
	% \item Тестируйте совместную значимость первых 7 коэффициентов ACF.
\end{enumerate}
\end{exercise}

\section{Стационарные ряды. Модель ARMA}

Во всех задачах по умолчанию уровень значимости 5\%.

% \subsection{Модель ARMA (Подгонка, выбор порядка, диагностика, прогнозирование)}

\subsection{Модель ARMA с константой}

\begin{exercise}
Пусть \(y_t\) -- логарифмическая доходность US GDP (\textbf{квартальные данные}) с 1990 по н.в.
\begin{enumerate}
	% \item Нарисуйте график ряда, ACF \& PACF для log-US GDP и для \(y_t\)
	\item Подгоните модель ARMA(1,1)
	\item Подгоните <<оптимальную>> модель ARMA и проведите её диагностику
	\item Постройте прогноз на 10 периодов
\end{enumerate}
\end{exercise}

\begin{exercise}
Пусть \(y_t\) -- логарифмическая доходность US M2 (\textbf{месячные данные}) с 1990 по н.в.
\begin{enumerate}
	% \item Нарисуйте график ряда, ACF \& PACF для log-US M2 и для \(y_t\)
	\item Подгоните модели
	\begin{align*}
		MA&(2) & AR&(2) & ARMA&(1,1) 
	\end{align*} 
	и проведите их диагностику
	\item Подгоните <<оптимальную>> модель ARMA и проведите её диагностику
	\item Постройте прогноз на 10 периодов
\end{enumerate}
\end{exercise}

\begin{exercise}
Пусть \(y_t\) -- логарифмическая доходность US M2 (\textbf{недельные данные}) с 1995 по н.в.
\begin{enumerate}
	% \item Нарисуйте график ряда, ACF \& PACF для log-US M2 и для \(y_t\)
	\item Подгоните модели
	\begin{align*}
		ARMA&(2,1) & ARMA&(1,2) & ARMA&(2,2)
	\end{align*} 
	и проведите их диагностику
	\item Подгоните <<оптимальную>> модель ARMA и проведите её диагностику
	\item Постройте прогноз на 10 периодов
\end{enumerate}
\end{exercise}

\begin{exercise}
Пусть \(y_t\) -- первая разность of 10-летней ставки (treasury securities  with constant maturity) 
(\textbf{квартальные данные}) с 1990 по н.в.
\begin{enumerate}
	% \item Нарисуйте график ряда, ACF \& PACF  для ставки и для \(y_t\)
	\item Подгоните модели
	\begin{align*}
		ARMA&(1,1) & ARMA&(2,1) & ARMA&(1,2) & ARMA&(2,2)
	\end{align*} 
	и проведите их диагностику
	\item Подгоните <<оптимальную>> модель ARMA и проведите её диагностику
	\item Постройте прогноз на 10 периодов
\end{enumerate}
\end{exercise}

\begin{exercise}
Пусть \(y_t\) -- первая разность of 10-летней ставки (treasury securities  with constant maturity) 
(\textbf{месячные данные}) с 1990 по н.в.
\begin{enumerate}
	% \item Нарисуйте график ряда, ACF \& PACF для ставки и для \(y_t\)
	\item Подгоните модели
	\begin{align*}
		ARMA&(1,1) & ARMA&(2,1) & ARMA&(1,2) & ARMA&(2,2)
	\end{align*} 
	и проведите их диагностику
	\item Подгоните <<оптимальную>> модель ARMA и проведите её диагностику
	\item Постройте прогноз на 10 периодов
\end{enumerate}
\end{exercise}

\begin{exercise}
Ряд \(y_t\) -- первая разность 3-месячной ставки (treasury bill, 
\textbf{квартальные данные}) с 1990 по н.в.
\begin{enumerate}
	% \item Нарисуйте график ряда, ACF \& PACF  для ставки и для \(y_t\)
	\item Подгоните модели
	\begin{align*}
		ARMA&(2,1) & ARMA&(1,2) & ARMA&(2,2)
	\end{align*} 
	и проведите их диагностику
	\item Подгоните <<оптимальную>> модель ARMA и проведите диагностику
	\item Постройте прогноз на 10 периодов
\end{enumerate}
\end{exercise}

\begin{exercise}
Ряд \(y_t\) -- первая разность 3-месячной ставки (treasury bill, 
\textbf{месячные данные}) с 1990 по н.в.
\begin{enumerate}
	% \item Нарисуйте график ряда, ACF \& PACF для ставки и для \(y_t\)
	\item Подгоните модели
	\begin{align*}
		ARMA&(2,1) & ARMA&(1,2) & ARMA&(2,2)
	\end{align*} 
	и проведите их диагностику
	\item Подгоните <<оптимальную>> модель ARMA и проведите диагностику
	\item Постройте прогноз на 10 периодов
\end{enumerate}
\end{exercise}

\subsection{Модель ARMA с трендом}

\begin{exercise}
Пусть \(y_t\) -- логарифм US GDP (\textbf{квартальные данные}) с 1990 по н.в.
\begin{enumerate}
	% \item Нарисуйте график ряда, ACF \& PACF для log-US GDP и для \(y_t\)
	\item Подгоните модель ARMA(1,1) с трендом и проведите её диагностику
	\item Подгоните <<оптимальную>> модель ARMA с трендом и проведите её диагностику
	\item Постройте прогноз на 10 периодов
\end{enumerate}
\end{exercise}

\begin{exercise}
Пусть \(y_t\) -- логарифм US M2 (\textbf{месячные данные}) с 1990 по н.в.
\begin{enumerate}
% \item Нарисуйте график ряда, ACF \& PACF для log-US M2 и для \(y_t\)
\item Подгоните модели 
	\begin{align*}
		MA&(2) & AR&(2) & ARMA&(1,1) 
	\end{align*} 
	с трендом и проведите их диагностику
	\item Подгоните <<оптимальную>> модель ARMA и проведите её диагностику
	\item Постройте прогноз на 10 периодов
\end{enumerate}
\end{exercise}

% \section{Модель ARMA}

% \begin{exercise}
% Для каждой модели ARMA укажите её порядок, проверьте её стационарность и 
% обратимость MA-части. В стационарном случае найдите \(\Exp x_t\). 
% Во всех моделях \(u_t\sim WN(0,\sigma^2)\)
% \begin{enumerate}
% 	\item \(x_t=10-0.99x_{t-1}+u_t+1.2u_{t-1}\)
% 	\item \(x_t=x_{t-1}-0.21x_{t-2}+u_t-u_{t-1}\)
% 	\item \(x_t=-3+3x_{t-1}-3x_{t-2}+x_{t-3}+u_t-2u_{t-1}+u_{t-2}\)
% 	\item \(x_t=2+2.2x_{t-1}-1.4x_{t-2}+0.2x_{t-3}+u_t-u_{t-1}-1.6x_{t-2}\)
% 	\item \(x_t=1.5+0.7x_{t-1}-0.11x_{t-2}+0.005x_{t-3}+u_t+2.6u_{t-1}-1.2u_{t-2}\)
% \end{enumerate}
% Напишите модель для \(\Delta x_t\) и \(\Delta^2 x_t\) 
% \end{exercise}

% \begin{exercise}
% По временному ряду длины $n=110$ были оценены следующие модели ARMA:
% \begin{enumerate}
% 	\item $y_t=16+0.67y_{t-1}+u_t-0.3u_{t-1}$, $s^2=1.63$;
% 	\item $y_t=15.1+0.6y_{t-1}-0.1y_{t-2}+u_t$, $s^2=1.8$;
% 	\item $y_t=17.1+0.71y_{t-1}+u_t-0.34u_{t-1}+0.12u_{t-2}$,
% 	$s^2=1.86$.
% \end{enumerate}
% Какую модель вы выберете? %для прогнозирования?
% \end{exercise}

% \begin{exercise}
% Для ряда из 200 наблюдения были оценены несколько ARMA-моде\-лей разного 
% порядка. В таблице приведены результаты оценки дисперсии остатков
% (т.е. $s^2$). Какой порядок модели можно назвать <<оптимальным>>?
% \begin{center}
% \begin{tabular}{l|l}
% 	порядок ARMA & $s^2$ \\ \hline
% 	(0,0) & 1.902 \\
% 	(1,0) & 1.662 \\
% 	(0,1) & 1.583 \\
% 	(1,1) & 1.442 \\ 
% 	(2,1) & 1.303 \\
% 	(1,2) & 1.423 \\
% 	(2,2) & 0.981 \\
% 	(3,2) & 0.972 \\
% 	(2,3) & 0.923 \\
% 	(3,3) & 0.893
% \end{tabular}
% \end{center}
% \end{exercise}

% \begin{exercise}
% Временной ряд $x_t$ -- первая разность логарифма реального ВВП США с 1984:2 по 2013:2. Для
% этого ряда были оценены следующие модели (в таблице приведены значения информационных
% критериев и $Q_{LB}$-статистики с лагом 4 для остатков):
% \begin{center}{\scriptsize
% \begin{tabular}{|l|c|c|c|c|c|c|c|}
% 	\hline
% 	модель & MA(1) & AR(1) & ARMA(1,1) & AR(2) & MA(2) & AR(3) & ARMA(2,1) \\ \hline
% 	AIC & -870.85  & -880.17 & -883.55 & -885.54 & -882.83 & -884.10 & -884.54 \\
% 	BIC & -862.57 & -871.88 & -872.50 & -874.49 & -871.78 & -870.29 & -870.73\\
% 	$Q_{LB}$ & 21.5127 & 11.0688 & 3.68299 & 1.34523 & 4.21655 & 0.963561 & 0.494841 \\
% 	\hline
% \end{tabular} }
% \end{center}
% Какая модель предпочтительней? Ответ обоснуйте.
% \end{exercise}

% \begin{exercise}
% По 100 наблюдениям была оценена модель 
% \[
% 	x_t=2+0.5x_{t-1}-0.06x_{t-2}+u_t.
% \] 
% и проверена её адекватность. Известно, что $x_{100}=3.1$, $x_{99}=2.5$,
% $x_{98}=2.8$, $x_{97}=3.6$. Найдите $\widehat{x}_{102}$,  $\widehat{x}_{103}$
% и $\widehat{x}_{104}$. 
% \end{exercise}
	
% \begin{exercise}
% Временной ряд $x_t$ -- первая разность логарифма реального ВВП США с 1984:2 по 2013:2
% (квартальные данные). Была оценена модель ARMA(2,1)
% \begin{center}
%  \begin{tabular}{|c|c|c|c|c|} 
% 	\hline
% 	& const & $x_{t-1}$ & $x_{t-2}$ & $u_{t-1}$  \\ \hline
% 	коэфф. & 0.007 & -0.026 & 0.397 & 0.378 \\ \hline
% 	$\Prob$-значение & $1.02\cdot 10^{-10}$ & 0.9172 & 0.0006 & 0.1534  \\
% 	\hline
% \end{tabular}
% \end{center}
% Известны значения временного ряда и остатки %оценённые значения ошибки 
% модели
% \begin{center}
% 	\begin{tabular}{|l|c|c|c|c|c|} \hline
% 	$t$ & 2012:2 & 2012:3 & 2012:4 & 2013:1 & 2013:2 \\ \hline
% 	$x_t$ & 0.003 & 0.007 & 0.0004 & 0.003 & 0.006 \\ \hline
% 	$\hat{u}_t$ & –0.007 & 0.001 & –0.006 & –0.002 & 0.003 \\
% 	\hline
%  \end{tabular}
% \end{center}
% Постройте прогноз на четвёртый квартал 2013 года (2013:4).
% \end{exercise}

% \section{Ряды с единичным корнем}

\section{Модель ARIMA}

% \subsection{ARIMA в Python}

\begin{exercise}
Пусть \(y_t\) -- логарифм US GDP (\textbf{квартальные данные}) с 1995 по н.в.
\begin{enumerate}
	\item Подгонка модели заданного порядка
	\begin{enumerate}
		\item Подгоните модели
		\begin{center}\small
		\begin{tabular}{l|c|c}
			Модель & drift/trend  & спецификация\\ \hline
			ARIMA(1,0,1) & + & \(y_t=\alpha_0+\alpha_1t+\phi y_{t-1}+u_t+\theta u_{t-1}\)\\
			ARIMA(1,1,0) & + & \(\diff y_t=\alpha_0+\phi\diff y_{t-1}+u_t+\theta u_{t-1}\)\\
			ARIMA(1,1,1) & - & \(\diff y_t=\phi\diff y_{t-1}+u_t+\theta u_{t-1}\) \\
			ARIMA(1,2,0) & - & \(\diff^2 y_t=\phi\diff^2 y_{t-1}+u_t\)\\ \hline
		\end{tabular}
		\end{center} 
		и постройте прогноз на 10 периодов. Значим ли снос/тренд?
		\item Проведите диагностику каждой модели.
		\item Проведите кросс-валидацию каждой модели. Какая предпочтительней?
	\end{enumerate}
	\item Примените тесты единичного корня и найдите порядок интегрирования для \(y_t\). 
	\item Подгонка <<оптимальной модели>>
	\begin{enumerate}
		\item Подгоните <<оптимальную>> модель ARIMA
		\item проведите её диагностику
		\item Постройте прогноз на 10 периодов
	\end{enumerate}
\end{enumerate}
\end{exercise}

% \begin{exercise}
% Пусть \(y_t\) -- логарифм US M2 (\textbf{квартальные данные}) с 1995 по н.в.
% \begin{enumerate}
% 	\item Примените тесты единичного корня и найдите порядок интегрирования для \(y_t\)
% 	Значим ли снос/линейный тренд?
% 	\item Подгоните модели (включение сноса зависит от предыдущего пункта)
% 	\begin{align*}
% 		ARIMA&(1,1,0) & ARIMA&(0,1,1) & ARIMA&(2,1,0) \\
% 		ARIMA&(0,1,2) & ARIMA&(1,2,0) & ARIMA&(0,2,1)
% 	\end{align*} 
% 	и проведите их диагностику	
% 	\item Подгоните <<оптимальную>> модель ARIMA и проведите её диагностику
% 	\item Постройте прогноз на 10 периодов
% \end{enumerate}
% \end{exercise}

\begin{exercise}
Пусть \(y_t\) -- логарифм US M2 (\textbf{месячные данные}) с 1995 по н.в.
\begin{enumerate}
	\item Подгонка модели заданного порядка
	\begin{enumerate}
		\item Подгоните модели
		\begin{center}
		\begin{tabular}{l|c}
			Модель & drift/trend \\ \hline
			ARIMA(2,0,2) & + \\
			ARIMA(2,1,0) & + \\
			ARIMA(2,1,1) & - \\
			ARIMA(1,2,0) & - \\ \hline
		\end{tabular}
		\end{center} 
		и постройте прогноз на 10 периодов. Значим ли снос/тренд?
		\item Проведите диагностику каждой модели.
		\item Проведите кросс-валидацию каждой модели. Какая предпочтительней?
	\end{enumerate}
	\item Примените тесты единичного корня и найдите порядок интегрирования для \(y_t\). 
	\item Подгонка <<оптимальной модели>>
	\begin{enumerate}
		\item Подгоните <<оптимальную>> модель ARIMA
		\item проведите её диагностику
		\item Постройте прогноз на 10 периодов
	\end{enumerate}
\end{enumerate}
\end{exercise}

\begin{exercise}
Пусть \(y_t\) -- логарифм US M2 (\textbf{недельные данные}) с 1995 по н.в.
\begin{enumerate}
	\item Подгонка модели заданного порядка
	\begin{enumerate}
		\item Подгоните модели
		\begin{center}
		\begin{tabular}{l|c}
			Модель & drift/trend \\ \hline
			ARIMA(3,0,2) & + \\
			ARIMA(2,1,0) & + \\
			ARIMA(2,1,1) & - \\
			ARIMA(2,2,0) & - \\ \hline
		\end{tabular}
		\end{center} 
		и постройте прогноз на 10 периодов. Значим ли снос/тренд?
		\item Проведите диагностику каждой модели.
		\item Проведите кросс-валидацию каждой модели. Какая предпочтительней?
	\end{enumerate}
	\item Примените тесты единичного корня и найдите порядок интегрирования для \(y_t\). 
	\item Подгонка <<оптимальной модели>>
	\begin{enumerate}
		\item Подгоните <<оптимальную>> модель ARIMA
		\item проведите её диагностику
		\item Постройте прогноз на 10 периодов
	\end{enumerate}
\end{enumerate}
\end{exercise}

\begin{exercise}
Пусть \(y_t\) -- 10-летняя ставка (treasury securities  with constant maturity 
\textbf{месячные данные}) с 2000 по н.в.
\begin{enumerate}
	\item Подгонка модели заданного порядка
	\begin{enumerate}
		\item Подгоните модели
		\begin{center}
		\begin{tabular}{l|c}
			Модель & drift/const \\ \hline
			ARIMA(2,0,2) & - \\
			ARIMA(2,0,2) & + \\
			ARIMA(2,1,0) & + \\
			ARIMA(2,1,1) & - \\
			ARIMA(2,2,0) & - \\ \hline
		\end{tabular}
		\end{center} 
		и постройте прогноз на 10 периодов. Значим ли снос/const?
		\item Проведите диагностику каждой модели.
		\item Проведите кросс-валидацию каждой модели. Какая предпочтительней?
	\end{enumerate}
	\item Примените тесты единичного корня и найдите порядок интегрирования для \(y_t\). 
	\item Подгонка <<оптимальной модели>>
	\begin{enumerate}
		\item Подгоните <<оптимальную>> модель ARIMA
		\item проведите её диагностику
		\item Постройте прогноз на 10 периодов
	\end{enumerate}
\end{enumerate}
\end{exercise}

\begin{exercise}
Пусть \(y_t\) -- 10-летняя ставка (treasury securities with constant matu\-ri\-ty) 
(\textbf{дневные данные}) с 2010 по н.в.
\begin{enumerate}
	\item Подгонка модели заданного порядка
	\begin{enumerate}
		\item Подгоните модели
		\begin{center}
		\begin{tabular}{l|c}
			Модель & drift/const \\ \hline
			ARIMA(3,0,2) & - \\
			ARIMA(3,0,2) & + \\
			ARIMA(3,1,0) & + \\
			ARIMA(3,1,1) & - \\
			ARIMA(2,2,0) & - \\ \hline
		\end{tabular}
		\end{center} 
		и постройте прогноз на 10 периодов. Значим ли снос/const?
		\item Проведите диагностику каждой модели.
		\item Проведите кросс-валидацию каждой модели. Какая предпочтительней?
	\end{enumerate}
	\item Примените тесты единичного корня и найдите порядок интегрирования для \(y_t\). 
	\item Подгонка <<оптимальной модели>>
	\begin{enumerate}
		\item Подгоните <<оптимальную>> модель ARIMA
		\item проведите её диагностику
		\item Постройте прогноз на 10 периодов
	\end{enumerate}
\end{enumerate}
\end{exercise}

\begin{exercise}
Пусть \(y_t\) -- 3-месячная ставки (treasury bill, \textbf{месячные данные}) с 2000 по н.в.
\begin{enumerate}
	\item Подгонка модели заданного порядка
	\begin{enumerate}
		\item Подгоните модели
		\begin{center}
		\begin{tabular}{l|c}
			Модель & drift/const \\ \hline
			ARIMA(2,0,2) & - \\
			ARIMA(2,0,2) & + \\
			ARIMA(2,1,0) & + \\
			ARIMA(2,1,1) & - \\
			ARIMA(2,2,0) & - \\ \hline
		\end{tabular}
		\end{center} 
		и постройте прогноз на 10 периодов. Значим ли снос/const?
		\item Проведите диагностику каждой модели.
		\item Проведите кросс-валидацию каждой модели. Какая предпочтительней?
	\end{enumerate}
	\item Примените тесты единичного корня и найдите порядок интегрирования для \(y_t\). 
	\item Подгонка <<оптимальной модели>>
	\begin{enumerate}
		\item Подгоните <<оптимальную>> модель ARIMA
		\item проведите её диагностику
		\item Постройте прогноз на 10 периодов
	\end{enumerate}
\end{enumerate}
\end{exercise}
	
\begin{exercise}
Пусть \(y_t\) -- 3-месячная ставки (treasury bill, \textbf{дневные данные}) с 2010 по н.в.
\begin{enumerate}
	\item Подгонка модели заданного порядка
	\begin{enumerate}
		\item Подгоните модели
		\begin{center}
		\begin{tabular}{l|c}
			Модель & drift/const \\ \hline
			ARIMA(3,0,2) & - \\
			ARIMA(3,0,2) & + \\
			ARIMA(3,1,0) & + \\
			ARIMA(3,1,1) & - \\
			ARIMA(2,2,0) & - \\ \hline
		\end{tabular}
		\end{center} 
		и постройте прогноз на 10 периодов. Значим ли снос/const?
		\item Проведите диагностику каждой модели.
		\item Проведите кросс-валидацию каждой модели. Какая предпочтительней?
	\end{enumerate}
	\item Примените тесты единичного корня и найдите порядок интегрирования для \(y_t\). 
	\item Подгонка <<оптимальной модели>>
	\begin{enumerate}
		\item Подгоните <<оптимальную>> модель ARIMA
		\item проведите её диагностику
		\item Постройте прогноз на 10 периодов
	\end{enumerate}
\end{enumerate}
\end{exercise}

\begin{exercise}
Пусть \(y_t\) -- логарифм S\&P500 (\textbf{дневные данные}) с 2010 по н.в.
\begin{enumerate}
	\item Подгонка модели заданного порядка
		\begin{enumerate}
			\item Подгоните модели
			\begin{center}
			\begin{tabular}{l|c}
				Модель & drift/const \\ \hline
				ARIMA(2,0,2) & - \\
				ARIMA(2,0,2) & + \\
				ARIMA(2,1,0) & + \\
				ARIMA(2,1,1) & - \\
				ARIMA(2,2,0) & - \\ \hline
			\end{tabular}
			\end{center} 
			и постройте прогноз на 10 периодов. Значим ли снос/const?
			\item Проведите диагностику каждой модели.
			\item Проведите кросс-валидацию каждой модели. Какая предпочтительней?
		\end{enumerate}
		\item Примените тесты единичного корня и найдите порядок интегрирования для \(y_t\). 
		\item Подгонка <<оптимальной модели>>
		\begin{enumerate}
			\item Подгоните <<оптимальную>> модель ARIMA
			\item проведите её диагностику
			\item Постройте прогноз на 10 периодов
		\end{enumerate}
	\end{enumerate}
	\end{exercise}

\begin{exercise}
Запишите спецификации следующих моделей
\begin{enumerate}
	\item ARIMA(0,1,1) без сноса и со сносом
	\item ARIMA(0,1,2) без сноса и со сносом
	\item ARIMA(1,1,0) без сноса и со сносом
	\item ARIMA(2,1,0) без сноса и со сносом
	\item ARIMA(0,2,0) без сноса и со сносом
	\item ARIMA(1,2,0) без сноса и со сносом
	\item ARIMA(0,2,1) без сноса и со сносом
\end{enumerate}
\end{exercise}

\section{Модель (*)ARCH}

\begin{exercise}
Пусть \(y_t\) -- лог-доходность US M2 (\textbf{недельные данные}) с 1995 по н.в.
\begin{enumerate}
	\item Подгоните модели AR-GARCH(p,o,q)
	\begin{center}
	\begin{tabular}{l|c}
		Модель & \(\lambda\) \\ \hline
		AR(1)-GARCH(1,0,1) & 2 \\
		AR(1)-GARCH(1,0,1) & 1 \\
		AR(2)-GARCH(1,0,1) & 2 \\
		AR(2)-GARCH(1,0,1) & 1 \\ \hline
	\end{tabular}
	\end{center} 
	и постройте прогноз на 10 периодов для ряда и его волатильности.
	\item Сравните модели по информационным критериям. Какая предпочтительней?
	\item Проведите кросс-валидацию моделей. Какая предпочтительней?
	\item Подгоните модели GARCH-in-Mean
	\begin{center}
		\begin{tabular}{l|c}
			Модель & \(f(x)\) \\ \hline
			AR(2)-GARCH(1,1) & \(\log x\) \\
			AR(2)-GARCH(1,1) & \(x\) \\
			AR(2)-GARCH(1,1) & \(\sqrt{x}\) \\
		\end{tabular}
	\end{center} 
	и постройте прогноз на 10 периодов для ряда и его волатильности.
	\item Сравните модели по информационным критериям. Какая предпочтительней?
\end{enumerate}
\end{exercise}

\begin{exercise}
Пусть ряд \(y_t\) -- первая разность 3-месячной ставки (treasury bill, \textbf{дневные данные}) с 2010 по н.в.
\begin{enumerate}
	\item Подгоните модели AR-GARCH(p,o,q)
	\begin{center}
	\begin{tabular}{l|c}
		Модель & \(\lambda\) \\ \hline
		AR(1)-GARCH(1,0,1) & 2 \\
		AR(1)-GARCH(1,0,1) & 1 \\
		AR(2)-GARCH(1,0,1) & 2 \\
		AR(2)-GARCH(1,0,1) & 1 \\ \hline
	\end{tabular}
	\end{center} 
	и постройте прогноз на 10 периодов для ряда и его волатильности.
	\item Сравните модели по информационным критериям. Какая предпочтительней?
	\item Проведите кросс-валидацию моделей. Какая предпочтительней?
	\item Подгоните модели GARCH-in-Mean
	\begin{center}
		\begin{tabular}{l|c}
			Модель & \(f(x)\) \\ \hline
			AR(1)-GARCH(1,1) & \(\log x\) \\
			AR(1)-GARCH(1,1) & \(x\) \\
			AR(1)-GARCH(1,1) & \(\sqrt{x}\) \\
		\end{tabular}
	\end{center} 
	и постройте прогноз на 10 периодов для ряда и его волатильности.
	\item Сравните модели по информационным критериям. Какая предпочтительней?
\end{enumerate}
\end{exercise}

\begin{exercise}
Пусть ряд \(y_t\) -- первая разность 10-летней ставки (treasury securities  with constant maturity, \textbf{дневные данные}) с 2010 по н.в.
\begin{enumerate}
	\item Подгоните модели AR-GARCH(p,o,q)
	\begin{center}
	\begin{tabular}{l|c}
		Модель & \(\lambda\) \\ \hline
		AR(1)-GARCH(1,0,1) & 2 \\
		AR(1)-GARCH(1,0,1) & 1 \\
		AR(2)-GARCH(1,0,1) & 2 \\
		AR(2)-GARCH(1,0,1) & 1 \\ \hline
	\end{tabular}
	\end{center} 
	и постройте прогноз на 10 периодов для ряда и его волатильности.
	\item Сравните модели по информационным критериям. Какая предпочтительней?
	\item Проведите кросс-валидацию моделей. Какая предпочтительней?
	\item Подгоните модели GARCH-in-Mean
	\begin{center}
		\begin{tabular}{l|c}
			Модель & \(f(x)\) \\ \hline
			AR(1)-GARCH(1,1) & \(\log x\) \\
			AR(1)-GARCH(1,1) & \(x\) \\
			AR(1)-GARCH(1,1) & \(\sqrt{x}\) \\
		\end{tabular}
	\end{center} 
	и постройте прогноз на 10 периодов для ряда и его волатильности.
	\item Сравните модели по информационным критериям. Какая предпочтительней?
\end{enumerate}
\end{exercise}


\begin{exercise}
Пусть ряд \(y_t\) -- лог-доходность S\&P500 (\textbf{дневные данные}) с 2010 по н.в.
\begin{enumerate}
	\item Подгоните модели
	\begin{center}
	\begin{tabular}{l|c}
		Модель & \(\lambda\) \\ \hline
		AR(1)-GARCH(1,0,1) & 2 \\
		AR(1)-GARCH(1,0,1) & 1 \\
		AR(1)-GARCH(1,1,1) & 2 \\
		AR(1)-GARCH(1,1,1) & 1 \\ \hline
	\end{tabular}
	\end{center} 
	и постройте прогноз на 10 периодов для ряда и его волатильности. 
	\item Сравните модели по информационным критериям. Какая предпочтительней?
	\item Проведите кросс-валидацию моделей. Какая предпочтительней?
	\item Подгоните модели GARCH-in-Mean
	\begin{center}
		\begin{tabular}{l|c}
			Модель & \(f(x)\) \\ \hline
			AR(1)-GARCH(1,1) & \(\log x\) \\
			AR(1)-GARCH(1,1) & \(x\) \\
			AR(1)-GARCH(1,1) & \(\sqrt{x}\) \\
		\end{tabular}
	\end{center} 
	и постройте прогноз на 10 периодов для ряда и его волатильности.
	\item Сравните модели по информационным критериям. Какая предпочтительней?
\end{enumerate}
\end{exercise}

% \begin{exercise}
% По 100 наблюдениям была оценена модель ARIMA(1,1,0) без сноса с параметрами
% \(\phi=0.8\) и проверена её адекватность. 

% Известно, что $x_{100}=8.2$, $x_{99}=7.9$,
% $x_{98}=6.7$, $x_{97}=7.2$. Найдите $\widehat{x}_{102}$,  $\widehat{x}_{103}$
% и $\widehat{x}_{104}$.
% \end{exercise}

% \begin{exercise}
% По 100 наблюдениям была оценена модель ARIMA(1,1,0) со сносом с параметрами
% \(\mu=0.2, \phi=-0.8\) и проверена её адекватность. 

% Известно, что $x_{100}=12.3$, $x_{99}=10.2$,
% $x_{98}=10.8$, $x_{97}=9.4$. Найдите $\widehat{x}_{102}$,  $\widehat{x}_{103}$
% и $\widehat{x}_{104}$.
% \end{exercise}

% \begin{exercise}
% По 100 наблюдениям была оценена модель ARIMA(1,1,1) без сноса с параметрами
% \(\phi=0.3, \theta=0.6\) и проверена её адекватность. 

% Известно, что $x_{100}=3.1$, $x_{99}=2.5$,
% $x_{98}=2.8$, $x_{97}=3.6$. Найдите $\widehat{x}_{102}$,  $\widehat{x}_{103}$
% и $\widehat{x}_{104}$.
% \end{exercise}

% \begin{exercise}
% По 100 наблюдениям была оценена модель ARIMA(0,2,0) без сноса и проверена её адекватность. 

% Известно, что $x_{100}=6.9$, $x_{99}=5.8$,
% $x_{98}=7.3$, $x_{97}=6.5$. Найдите $\widehat{x}_{102}$,  $\widehat{x}_{103}$
% и $\widehat{x}_{104}$.
% \end{exercise}

% \begin{exercise}
% По 100 наблюдениям была оценена модель ARIMA(1,2,0) без сноса
% с параметрами \(\phi=0.6\) и проверена её адекватность. 

% Известно, что $x_{100}=6.9$, $x_{99}=5.8$,
% $x_{98}=7.3$, $x_{97}=6.5$. Найдите $\widehat{x}_{102}$,  $\widehat{x}_{103}$
% и $\widehat{x}_{104}$.
% \end{exercise}

% \section{Многомерные ряды. Коинтеграция}

\section{Многомерные ряды. Модель VAR/VECM. Коинтеграция}

\begin{exercise}
Рассмотрим \textbf{недельные} данные с 2000 г по н.в. по следующим переменными
\begin{itemize}
	\item первая разность 3-месячной ставки (3-Month Treasury Bill)
	\item первая разность 6-месячной ставки (6-Month Treasury Bill)
	\item первая разность 10-летней ставки (Treasury Securities at 10-Year Constant Maturity)
\end{itemize}
Сформируйте многомерный ряд и визуализируйте его.
\begin{enumerate}
	\item Фиксированный порядок
	\begin{enumerate}
		\item Подгоните модели VAR(1), VAR(2), VAR(3)
		\item постройте прогноз на 10 периодов по каждой модели
		\item Проведите кросс-валидацию моделей. Какая предпочтительней?
	\end{enumerate}
	\item <<Оптимизация>> порядка
	\begin{enumerate}
		\item Подгоните модель VAR <<оптимального>> порядка
		\item Проведите её диагностику
		\item Постройте прогноз на 5 периодов. Постройте FEVD
		\item Постройте IRF, использую исходное упорядочивание переменных
		\item Проведите тест Гренджера на причинность
	\end{enumerate}
\end{enumerate}
\end{exercise}

\begin{exercise}
Рассмотрим \textbf{дневные} данные с 2000 г по н.в. по следующим переменными
\begin{itemize}
	\item первая разность 3-месячной ставки (3-Month Treasury Bill)
	\item первая разность 6-месячной ставки (6-Month Treasury Bill)
	\item первая разность 10-летней ставки (Treasury Securities at 10-Year Constant Maturity)
\end{itemize}
Сформируйте многомерный ряд и визуализируйте его.
\begin{enumerate}
	\item Фиксированный порядок
	\begin{enumerate}
		\item Подгоните модели VAR(1), VAR(2), VAR(3)
		\item постройте прогноз на 10 периодов по каждой модели
		\item Проведите кросс-валидацию моделей. Какая предпочтительней?
	\end{enumerate}
	\item <<Оптимизация>> порядка
	\begin{enumerate}
		\item Подгоните модель VAR <<оптимального>> порядка
		\item Проведите её диагностику
		\item Постройте прогноз на 5 периодов. Постройте FEVD
		\item Постройте IRF, использую исходное упорядочивание переменных
		\item Проведите тест Гренджера на причинность
	\end{enumerate}
\end{enumerate}
\end{exercise}

\begin{exercise}
Рассмотрим \textbf{месячные} данные с 1995 г по н.в. по следующим переменными
\begin{itemize}
	\item первая разность 3-месячной ставки (3-Month Treasury Bill)
	\item первая разность 6-месячной ставки (6-Month Treasury Bill)
	\item первая разность 10-летней ставки (Treasury Securities at 10-Year Constant Maturity)
	\item лог-доходность US M2
\end{itemize}
Сформируйте многомерный ряд и визуализируйте его.
\begin{enumerate}
	\item Фиксированный порядок
	\begin{enumerate}
		\item Подгоните модели VAR(1), VAR(2), VAR(3)
		\item постройте прогноз на 10 периодов по каждой модели
		\item Проведите кросс-валидацию моделей. Какая предпочтительней?
	\end{enumerate}
	\item <<Оптимизация>> порядка
	\begin{enumerate}
		\item Подгоните модель VAR <<оптимального>> порядка
		\item Проведите её диагностику
		\item Постройте прогноз на 5 периодов. Постройте FEVD
		\item Постройте IRF, использую исходное упорядочивание переменных
		\item Проведите тест Гренджера на причинность
	\end{enumerate}
\end{enumerate}
\end{exercise}

\begin{exercise}[VECM]
Рассмотрим \textbf{недельные} данные с 2005 г по н.в. по следующим переменными
\begin{itemize}
	\item 3-месячная ставки (3-Month Treasury Bill)
	\item 6-месячная ставки (6-Month Treasury Bill)
	\item 1-летняя ставка (Treasury Securities at 1-Year Constant Maturity)
	\item 10-летняя ставка (Treasury Securities at 10-Year Constant Maturity)
\end{itemize}
Сформируйте многомерный ряд и визуализируйте его.
\begin{enumerate}
	\item Найдите ранг коинтеграции
	\item Оцените модель VECM <<оптимального>> порядка
	\item Проведите её диагностику
	\item Постройте прогноз на 5 периодов. Постройте FEVD
	\item Постройте IRF, использую исходное упорядочивание переменных
	\item Проведите тест Гренджера на причинность
\end{enumerate}
\end{exercise}

\begin{exercise}[VECM]
Рассмотрим \textbf{недельные} данные с 2005 г по н.в. по следующим переменными
\begin{itemize}
	\item 3-месячная ставки (3-Month Treasury Bill)
	\item 6-месячная ставки (6-Month Treasury Bill)
	\item 1-летняя ставка (Treasury Securities at 1-Year Constant Maturity)
	\item 10-летняя ставка (Treasury Securities at 10-Year Constant Maturity)
\end{itemize}
Сформируйте многомерный ряд и визуализируйте его.
\begin{enumerate}
	\item Найдите ранг коинтеграции
	\item Оцените модель VECM <<оптимального>> порядка
	\item Проведите её диагностику
	\item Постройте прогноз на 5 периодов. Постройте FEVD
	\item Постройте IRF, использую исходное упорядочивание переменных
	\item Проведите тест Гренджера на причинность
\end{enumerate}
\end{exercise}

\begin{exercise}[VECM]
Рассмотрим \textbf{месячные} данные с 1995 г по н.в. по следующим переменными
\begin{itemize}
	\item 3-месячная ставки (3-Month Treasury Bill)
	\item 6-месячная ставки (6-Month Treasury Bill)
	\item 1-летняя ставка (Treasury Securities at 1-Year Constant Maturity)
	\item 10-летняя ставка (Treasury Securities at 10-Year Constant Maturity)
	\item лог-M2
\end{itemize}
Сформируйте многомерный ряд и визуализируйте его.
\begin{enumerate}
	\item Найдите ранг коинтеграции
	\item Оцените модель VECM <<оптимального>> порядка
	\item Проведите её диагностику
	\item Постройте прогноз на 5 периодов. Постройте FEVD
	\item Постройте IRF, использую исходное упорядочивание переменных
	\item Проведите тест Гренджера на причинность
\end{enumerate}
\end{exercise}

% \begin{exercise}
% Рассмотрим многомерный ряд \texttt{Canada} из пакета \texttt{vars}.
% \begin{enumerate}
% 	\item нарисуйте график исходных уровней ряда и первой разности 
% 	\item подгоните VAR модель для первой разности
% 	\item тестируйте её на серийную корреляцию
% 	\item постройте прогноз на 5 периодов. Постройте FEVD
% 	\item Постройте IRFs использую исходное упорядочивание переменных
% 	\item Проведите тест Гренджера на причинность
% \end{enumerate}
% \end{exercise}

% \begin{exercise}
% Рассмотрим многомерный ряд для ставок: 3-month, 1-year, 10-year (месячные данные), 
% с 2000 по н.в.)
% \begin{enumerate}
%	\item нарисуйте график исходных уровней ряда и первой разности
% 	\item Найдите порядок коинтеграции
% 	\item подгоните VAR модель для уровней ряда. 
% 	\item подгоните VAR модель для первой разности. 
% 	\item тестируйте модель на серийную корреляцию 
% 	\item постройте прогноз на 5 периодов для обоих моделей. Постройте FEVD
% 	\item постройте IRFs использую исходное упорядочивание переменных
% 	\item Проведите тест Гренджера на причинность
% \end{enumerate}
% \end{exercise}

% \begin{exercise}
% Рассмотрим многомерный ряд для ставок: 3-month, 10-year, AAA, BAA (месячные данные), 
% с 2000 по н.в.)
% \begin{enumerate}
% 	\item нарисуйте график исходных уровней ряда и первой разности
% 	\item Найдите порядок коинтеграции
% 	\item подгоните VAR модель для уровней ряда. 
% 	\item подгоните VAR модель для первой разности. 
% 	\item тестируйте модель на серийную корреляцию 
% 	\item постройте прогноз на 5 периодов для обоих моделей. Постройте FEVD
% 	\item постройте IRFs использую исходное упорядочивание переменных
% 	\item Проведите тест Гренджера на причинность
% \end{enumerate}
% \end{exercise}

% \section{Модель VAR}

\begin{exercise}
Рассмотрим VAR(1)
\begin{align*}
	\vectx_t&=\matrixA\vectx_{t-1}+\vectu_t &
	\vectx_t&=\begin{pmatrix} x_t \\ y_t \end{pmatrix} &
	\vectu_t&=\begin{pmatrix} u_t \\ v_t \end{pmatrix}
\end{align*}
где 
\begin{align*}
	\vectu_t&\sim WN(0,\Sigma) &
	\Sigma&=\begin{pmatrix}
	\sigma^2_u & \sigma_{uv} \\ \sigma_{uv} & \sigma^2_v
	\end{pmatrix}>0
\end{align*}
т.е. $u_t\sim WN(0,\sigma_u^2)$,   $v_t\sim WN(0,\sigma_v^2)$,
$\cov(u_t,v_t)=\sigma_{uv}$. 
	
Проверить условие стационарности для следующих матриц
\begin{align*}
	\matrixA&=\begin{pmatrix} 0.5 & 1 \\ 0 & 0.3 \end{pmatrix} &
	&\begin{pmatrix} 0 & 0.5 \\ -0.5 & 0 \end{pmatrix} &
	& \begin{pmatrix} 1 & 3 \\ 0 & 0.2  \end{pmatrix} &
	& \begin{pmatrix} 0 & 1 \\ 0 & 1  \end{pmatrix} &
	& \begin{pmatrix} 1 & 1 \\ 1 & 1  \end{pmatrix} &
	& \begin{pmatrix} 0 & 1 \\ 1 & 1  \end{pmatrix}
\end{align*}
\end{exercise}
	
\begin{exercise}
Рассмотрим VAR-модели
\begin{align*}
	& \begin{cases} x_t=x_{t-1}+u_t \\ y_t=y_{t-1}+v_t \end{cases} &
	& \begin{cases} x_t=x_{t-1}+u_t \\ y_t=cx_{t}+v_t \end{cases} \\
	& \begin{cases} x_t=x_{t-1}+u_t \\ y_t=x_{t}+x_{t-1}+v_t \end{cases} &
	& \begin{cases} x_t=3x_{t-1}-7y_{t-1}+u_t \\ y_t=x_{t-1}-2.5y_{t-1}+v_t \end{cases}
\end{align*}
\begin{enumerate}
	\item Запишите в матричном виде.
	\item Проверить условие стационарности.
	\item Какие ряды коинтегрированы?
	\begin{itemize}
		\item Если ряды коинтегрированы, то запишите VECM модель
		и найдите коинтеграционные соотношения
		\item Если ряды не коинтегрированы, то запишите VAR-модель для дифференцированных рядов.
	\end{itemize}
\end{enumerate}
\end{exercise}
	
\begin{exercise}
Рассмотрим $\VAR(1)$
\begin{align*}
	\vectx_t&=\matrixA\vectx_{t-1}+\vectu_t &
	\vectx_t&=\begin{pmatrix} x_t \\ y_t \\ z_t \end{pmatrix} &
	\vectu_t&=\begin{pmatrix} u_t \\ v_t \\ w_t \end{pmatrix}\sim WN(0,\Sigma)
\end{align*}
Проверьте условие стационарности для матриц
\begin{align*}
	\matrixA&=\begin{pmatrix}
	0 & 1 & 3 \\ -1 & 0 & -2 \\ 0 & 0 & 0.5 \end{pmatrix} &
	&\begin{pmatrix}
		0 & 0 & 0.5 \\ 0.5 & 0 & 0 \\0 & 0.5 & 0
	\end{pmatrix}
\end{align*}
\end{exercise}
	
\begin{exercise}
Рассмотрим $\VAR(2)$
\begin{align*}
	\vectx_t&=\matrixA_1\vectx_{t-1}+\matrixA_2\vectx_{t-2}+\vectu_t &
	\vectx_t&=\begin{pmatrix} x_t \\ y_t \end{pmatrix} &
	\vectu_t&=\begin{pmatrix} u_t \\ v_t \end{pmatrix}\sim
	WN(0,\Sigma)
\end{align*}
Проверьте условие стационарности для матриц
\begin{align*}
	1)\,\,\matrixA_1&=\begin{pmatrix} 2 & 3 \\ 0 & 1 \end{pmatrix} &
	\matrixA_2&=\begin{pmatrix} -1 & 3 \\ 0 & -0.25 \end{pmatrix} \\
	2)\,\,\matrixA_1&=\begin{pmatrix} 0 & 0.5 \\ 0.5 & 0 \end{pmatrix} &
	\matrixA_2&=\begin{pmatrix} 0 & -0.25 \\ -0.25 & 0 \end{pmatrix}
\end{align*}
\end{exercise}
	
\begin{exercise}
Рассмотрим модели
\begin{align*}
	& \begin{cases} x_t=2x_{t-1}-x_{t-2}+u_t \\ y_t=1.5y_{t-1}-0.5y_{t-2}+v_t \end{cases} \\
	& \begin{cases} x_t=1.5x_{t-1}+y_{t-1}-0.5x_{t-2}-y_{t-2}+u_t \\
	y_t=-x_{t-1}-0.5y_{t-1}+x_{t-2}+1.5y_{t-2}+v_t \end{cases} \\
	& \begin{cases} x_t=x_{t-1}+u_t \\ y_t=x_{t}+x_{t-1}+v_t \\ z_t= x_{t}+y_{t-1}+w_t \end{cases} \\
	& \begin{cases} x_t=y_{t-1}+u_t \\ y_t=z_{t-1}+v_t \\ z_t= x_{t-1}+w_t \end{cases}
\end{align*}
\begin{enumerate}
	\item Запишите в матричном виде.
	\item Проверить условие стационарности.
	\item Какие ряды коинтегрированы?
	\begin{itemize}
		\item Если ряды коинтегрированы, то запишите VECM модель
		и найдите коинтеграционные соотношения
		\item Если ряды не коинтегрированы, то запишите VAR-модель для дифференцированных рядов.
	\end{itemize}
\end{enumerate}
\end{exercise}

\newpage

\appendix

\section{Библиотеки Python}

\begin{table}[h]%\footnotesize
	\begin{tabular}{l|l}
		Библиотека & Описание \\ \hline
		\texttt{pandas} & Табличные данные \\
		& (кросс-секции, панели, временные ряды) \\
		\texttt{numpy} & Работа с массивами, преобразование данных \\
		\texttt{yfinance} & Загрузка данных с \textcolor{blue}{finance.yahoo.com}\\
		\texttt{pandas-datareader} & Загрузка данных из внешних БД \\
		&  (FRED, \textcolor{blue}{finance.yahoo.com} etc) \\
		\texttt{statsmodels} & Регрессионный анализ, \\
		& базовые модели временных рядов \\
		\texttt{arch} & Тесты и модели временных рядов \\
		\texttt{pmdarima} & ARIMA-модель \\
		\texttt{scikit-learn} & Методы машинного обучения \\
		\texttt{sktime} & анализ временных рядов и ML\\ 
		\texttt{scipy.stats} & Статистические методы (распределения и др) \\
		\texttt{seaborn} & Визуализация статистических данных \\
		\texttt{matplotlib} & Визуализация данных \\
		\texttt{plotly} & Визуализация данных \\
		\hline
	\end{tabular}
	\caption{Основные библиотеки Python для анализа временных рядов}
\end{table}


\end{document}
